\documentclass{winnower}
\usepackage{indentfirst}
\usepackage{graphicx}
\usepackage{caption}
\usepackage{subfigure}
\usepackage{xcolor}
\usepackage{float}
\usepackage[section]{placeins}
\usepackage{multirow}
\usepackage{booktabs}
\setlength{\belowcaptionskip}{-0.5cm}

\begin{document}

\title{Homework1}

\author{Haoyu Guan}

\affil[]{Questrom School of Business, Boston University}







\date{2020.02.11}

\maketitle




%-------------------------------------------------%
\section{Exercise 1}
%-------------------------------------------------%

One extension of the model of the double spending problem for
the blockchain is to allow the attacker to give up after the attacked is n blocks
behind. This is related to the following problem. Suppose that the gambler
continues to bet until either wins n dollars or loses m dollars. What is the
probability that the gambler quits as a winner?


\newpage

\section{Exercise 2}

One extension of the model of the double spending problem
for the blockchain is to allow the winning probability depending on the state
variable. This is related to the following problem. Suppose in the gambler��s ruin
problem that the probability of winning depending on the gambler��s current
fortune, i.e. pj is the probability that the gambler wins a bet when the wealth
is j. Compute $Q_i$

\end{document}
